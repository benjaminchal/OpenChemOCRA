
\chapter{Core organic chemistry}
This module introduces organic chemistry and its important applications to everyday life, including current environmental concerns associated with sustainability.

\section{The basics 4.1.1}

	\paragraph{IUPAC rules} on the nomenclature of organic compounds is as follows,
		
	Firstly let me explain the parts of the name. Stem, prefix and suffix. The stem refers to the number of carbon atoms in the longest chain (just remember Monkeys Eat Peanut Butter + Greek); the prefix is in reference to a side chain (or a functional group); the suffix is added to indicate functional groups.
	
	We have three types of hydrocarbons, aliphatic, acyclic and aromatic. We will mostly focus on aliphatic hydrocarbons of which there are three homologous\footnote{A group of chemicals with the same functional group, differing by an addition of CH$_2$} series alkanes, alkenes and alkynes.
	
	\paragraph{Aliphatic} hydrocarbons are one which are joined in chains e.g,
	\begin{center}
		\chemfig{H-C(-[:90]H)(-[:270]H)-C(-[:90]CH_3)(-[:270]H)-C(-[:90]H)(-[:270]H)-H}
		
		\textit{2-methylethane}
	\end{center}
	
	\begin{samepage}
	\paragraph{Alicyclic} hydrocarbons join in cyclic shapes and look like this,
	\begin{center}
		\chemfig{*6(------)}
		
		\textit{Cyclohexane}
	\end{center}
	\end{samepage}
	
	\paragraph{Aromatic} hydrocarbons have benzene rings which contain all or some of the carbon atoms,
	\begin{center}
		\chemfig{\chemfig{**6(------)}}
		
		\textit{Benzene}
	\end{center}
	
	\paragraph{Functional groups (prefixes)} are as follows,

	\begin{center}		
	\begin{tabular}{c|c|c|c}
Functional group&Prefix&Suffix&Displayed \\ \hline \hline
carboxylic acids&none&-oic acid&\chemfig{-C(=[1]O)(-[7]OH)}\\ \hline
aldehydes&none&-al&\chemfig{-C(=[1]O)(-[7]H)}\\ \hline
ketones&none&-one&\chemfig{C-C(=[:90]O)-C}\\ \hline
alchols&hydroxy-&-ol&\chemfig{-OH}\\ \hline
amines&amino-&-amine&\\ \hline
ethers&alkoxy-&-ether&\\ \hline
fluorine&fluoro-&none&\\ \hline
chlorine&chloro-&none&\\ \hline
bromine&bromo-&none&\\ \hline
iodine&iodo-&none&\\ \hline
	\end{tabular}
	\end{center}
	
	I included a few more, I should imagine you only need the first 3 and the haloalkane ones (given that they are the ones in the textbook).
	
	\paragraph{Displaying formula} There are three ways to display chemical formula (for the exam). These being the following,
	
	\textit{General formula} is the simplest algebraic formula of a member of a homologous series. For example, for an alkane C$_n$H$_{2n+2}$.
	
	\textit{Structural formula} is the minimal detail that shows the arrangement of atoms in a molecule. We write out each alkile group\footnote{C atoms in the main chain} and there surrounding atoms. E.g. for 2,4-pentandione we have \ch{CH3COCH2COCH3}.
	
	\textit{Displayed formula} is where the full molecule is drawn so in the case of 2,4-pentandione,
	
	\begin{center}
		\chemfig{H-C(-[:90]H)(-[:270]H)-C(=[:90]O)-C(-[:90]H)(-[:270]H)-C(=[:90]O)-C(-[:90]H)(-[:270]H)-H}
	\end{center}
	
	\textit{Skeletal formula} is the simplified organic formula, shown by removing hydrogen atoms from alkyl chains, leaving just a carbon skeleton and associated functional groups. So in the case of 2,4-pentandione,
	
	\begin{center}
		\chemfig{[:-35.25]-[:35.25](=[:90]O)-[:-35.25]-[:35.25](=[:90]O)-}
	\end{center}
	
	\paragraph{Definition} time (and summary, straight from the horse's mouth),
	\begin{itemize}
		\item homologous series (a series of organic compounds having the same functional group but with each successive member differing by CH$_2$)
		\item functional group (a group of atoms responsible for the characteristic reactions of a compound)
		\item alkyl group (of formula C$_n$H$_{2n+1}$)
		\item aliphatic (a compound containing carbon and hydrogen joined together in straight chains, branched chains or non-aromatic rings)
		\item alicyclic (an aliphatic compound arranged in non-aromatic rings with or without side chains)
		\item aromatic (a compound containing a benzene ring)
		\item saturated (single carbon–carbon bonds only) and unsaturated (the presence of multiple carbon–carbon bonds, including C=C, C/C and  aromatic rings)
	\end{itemize}
	
	\paragraph{Structural isomers} are compounds with the same molecular formula but different structural formulae. Take, for \ch{C4H10}. Sound simple,
	\begin{center}
		\chemfig{-[:-35.25]-[:35.25]-[:-35.25]}
	\end{center}
	But now imagine a possible combination that could share the same molecular formula,
	\begin{center}
		\chemfig{-[:-30](-[:-90])-[:30]}
	\end{center}
	This is called a structural isomer. Both 2-methylpropane and butane are \ch{C4H10}. This is where structural formula and skeletal diagrams come in handy.
	
\section{Reaction Mechanisms 4.1.1}

	\paragraph{Homolytic fission} is where a covalent bond breaks and the electrons are shared evenly with each bonding 
atom receiving one electron from the bonded pair, forming two radicals.

	\paragraph{Heterolytic fission} is where a covalent bond breaks and the electrons are not shared evenly with one bonding atom receiving both electrons from the bonded pair.
	
	\paragraph{Radicals} are a species with an unpaired electron. We use a dot to show this, e.g. \ch{H3C-CH3 -> H3C "$\bullet$" {} + CH3 "$\bullet$" {}}. This being homolytic fission as each bonded atom receives one electron.
	
	\paragraph{Curly arrows} are used to describe the movement of an electron pair. They show either heterolytic fission or the formation of a covalent bond.
	
	\paragraph{Reaction mechanism diagrams} are used to show reaction mechanisms. They need to be sufficiently detailed, to show clearly the movements of an electron pair, with curly arrows and relevant dipoles.
	
\section{Alkanes 4.1.2}
	
	\paragraph{Alkanes} are saturated hydrocarbons containing single C–C and C–H bonds as $\sigma$-bonds (overlap of orbitals directly between the bonding atoms); free rotation of the $\sigma$-bond.
	
	\paragraph{A $\sigma$-bond} is the result of two overlapping orbital in. It is a single covalent bond.
	
	\paragraph{The bond angle} formed when a carbon atom has 4 $\sigma$-bonds is 109.5\degree .
	
	\paragraph{Trends in boiling points} are also obvious. The larger the molecule the more London forces there are at play. This means that the larger the molecule the more energy is needed to break the intermolecular forces and hence boiling the alkane. We also see the boiling point lower if the molecules branch. This is due to less surface area.
	
	\paragraph{Reactivity} of alkanes is low. This is because of the relative stability\footnote{High bond enthalpy} and very low polarity of the $\sigma$-bonds. 
	
	\paragraph{The combustion of alkanes} is seen everywhere. This is because fossil fuels like methane are alkanes. The longer the chain the more energy released per mole. The combustion reaction is as standard producing \ch{CO2} and \ch{H2O}.
	
	\paragraph{Alkanes and halogens} can react in the presence of UV radiation,
	\begin{center}
		\ch{CH4(g) + Br2(l) ->[UV] CH3Br(g) + HBr(g)}
	\end{center}
	However you need to know the mechanism of this reaction. It is as follows:
	\begin{enumerate}
		\item First the bond in the bromine molecule is broken up by homolytic fission \ch{Br-Br ->[UV] Br "$\bullet$" {} + "$\bullet$" {} Br}
		\item Next we have the propagation steps. This forms a chain reaction,
		\begin{enumerate}
			\item \ch{CH4 + Br "$\bullet$" {} -> "$\bullet$" {} + HBr}
			\item \ch{"$\bullet$" {} CH3 + Br2 -> CH3Br + Br "$\bullet$" {}}
		\end{enumerate}
		\item Finaly there is the termination, in which the two radicals collide forming a molicule with all the electron pairs. There are a number of ways this could happen.
		\begin{itemize}
			\item \ch{Br "$\bullet$" {} + "$\bullet$" {} Br -> Br2}
			\item \ch{"$\bullet$" {} CH3 + "$\bullet$" {} CH3 -> C2H6}
			\item \ch{"$\bullet$" {} CH3 + "$\bullet$" {} Br -> CH3Br}
		\end{itemize}
	\end{enumerate}
	
\section{Basics in Alkenes 4.1.3}

	\paragraph{Alkenes} are unsaturated hydrocarbons containing a C=C bond comprising a $\pi$-bond (sideways overlap of adjacent p-orbitals above and below the bonding C atoms) and a $\sigma$-bond (overlap of orbitals directly between the bonding atoms). The $\pi$ bonds prevent the C=C bond from rotating freely. I will explain later.
	
	\paragraph{The structure} is trigonal planar with the bonding angle around each C=C alkenes is 120\degree .
	
	\paragraph{Stereoisomers} are compounds with the same structural formula but with a different arrangement in space. We look at E/Z isomerism. This is when we see stereoisomers are formed due to a restriction of rotation around the C=C group.
	
	\paragraph{cis–trans isomerism} special case of E/Z isomerism in which two of the substituent groups attached to each carbon atom of the C=C group are the same.
	
	\paragraph{Cahn–Ingold–Prelog (CIP) priority rules} are used to strictly define EZ isomers. It is based on 'priority' of the atoms bonded to the C=Cs. The 'priority' is down to the relative height of atomic number.
	\begin{center}
		\chemfig{C(-[:120]CH_3)(-[:-120]H)=C(-[:60]H)(-[:-60]CH_3)}
	\end{center}
	Will be an E isomer because C has a higher atomic number than H.
	\begin{center}
		\chemfig{C(-[:120]H)(-[:-120]CH_3)=C(-[:60]H)(-[:-60]CH_3)}
	\end{center}
	And be a Z isomer.
	
\section{Reactions of Alkenes 4.1.3}

	\paragraph{The $\pi$-bond} has a low bond enthalpy. This makes alkenes far more reactive than alkanes because it is so easy to break the $\pi$-bond
	
	\paragraph{Addition reactions} are reactions in which add atoms to an alkene by breaking the double bond. Here are the reactions (remember them),
	
	\begin{itemize}
		\item \ch{CH_{2n} + H2 ->[Ni] CH_{2n+2}} is called hydrogenation reactions.
		\item \ch{CH_{2n} + X2 -> CH_{2n}X2} where X is a halogen. This is a common test for unsaturated hydrocarbons. We add a halogen like bromine and if the solution goes clear it is saturated and a dihaloalkanes has been formed
		\item \ch{CH_{2n} + HX -> CH_{2n+1}X} where X is a halogen. This reaction forms haloalkanes.
		\item \ch{CH_{2n} + H2O(g) ->[H "$_3$" {} PO "$_4$" {}] CH_{2n}OH} is called a hydration reactions. We need an acid catalyst and it forms alcohols.
	\end{itemize}
	
	%INSERT DIAGRAMS FOR ELECTROPHILIC ADDITION
	
	\paragraph{Markownikoff's rule} is used to predict the formation of a major organic predict in addition reactions of H-X to unsymmetrical alkenes, e.g H-Br to propene.
	
	To do this we look at the number of hydrogen atoms attached to the carbon atom. The more positive (H$^{\delta +}$ will be attracted to the carbon on the side of the double bond with the most hydrogen. For example,
	
	\begin{center}
		\chemfig{H-C(-[:90]H)=C(-[:90]H)-C(-[:90]H)(-[:-90]H)-H} + \chemfig{H$^{\delta +}$-Br$^{\delta -}$} \longrightarrow
		
		\vspace{7mm}
		
		\chemfig{H-C(-[:90]H)(-[:-90]H)-C(-[:90]H)(-[:-90]\oplus)-C(-[:90]H)(-[:-90]H)-H} + \ch{Br "$^{\oplus}$" {}} \longrightarrow
		
		\vspace{7mm}
		
		\chemfig{H-C(-[:90]H)(-[:-90]H)-C(-[:90]H)(-[:-90]Br)-C(-[:90]H)(-[:-90]H)-H}
		
		\vspace{7mm}
		
		\textit{When drawing in exams use curly arrows} %ADD CURLY ARROWS TO EQUATIONS IF TIME.
	\end{center}
	
	\paragraph{Polymerisation} remains the same. This time you need to know how to determine n from the number of monomers. Just rearrange the question into the following and it is easy,
	
	\begin{center}
		\chemfig{C(-[:120]W)(-[:-120]Y)=C(-[:60]X)(-[:-60]Y)}
	\end{center}
	
	\paragraph{Environmental impact} of waste polymers is a serious issue. To deal with waste polymers we can do the following,
	\begin{enumerate}
		\item We can use burn waste polymers for energy production.
		\item We can use 'Feedstock recycling' to reclaim monomers, gases or oil from waste polymers.
		\item We can simply sort and recycle polymers to be used again.
	\end{enumerate}
	The concern when burning polymers are the toxic gasses that are produced. An example of this id the combustion of halogenated plastics forming HCl. HCl is toxic so needs to be removed. %MAY NEED EXPANDING
	
	\paragraph{Biodegradable polymers} are ones which can be broken down by micro-organisms into water and CO$_2$. These polymers are usually made from starch or cellulose and the advantages to using them are obvious. When burred in a hole (a landfill) they will be less dangerous and degrade quickly. This will help save the environment.
	
	\paragraph{Photodegradable polymers} are much the same. Except this time it is light that we use to break down the polymer. These are oil-based polymers which contain bonds that weaken when they absorb light. This kick-starts the degradation. Alternatively light absorbing additives can be used.
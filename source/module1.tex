\chapter{Practical skills in chemistry}
	Practical skills are crucial to chemistry.
	Chemistry is a practical subject; as such we should understand how to correctly carry out our experiments.
	Practical skills are a recurring theme throughout the other 4 modules so a good knowledge of these skills is vital to success.

\section{Practical stuff 1.1.1 \& 1.1.2}
	\paragraph{Planning} is crucial to any and all experiment doing.
	In this course you should be able to do several things with ease.
	
	Firstly you should understand the range of equipment available and there appropriate usage.
	You should then have the mental capacity to apply this knowledge to a given situation and, in so doing, design a practical.
	This understanding comes with time and I will not go into much detail.
	
	Secondly you should understand these distinctions.
	\textbf{Independent} variables are ones which the experimenter changes e.g. the temperature of the heater.
	\textbf{Dependent} variables depend on the independent ones and are measured e.g. the amount of gas released from the reaction.
	\textbf{Control} variables are ones which remain constant and are controlled by the experimenter e.g. the amount of substance added.
	
	Because I like maths think of it like this. Let y be the dependent, x be the independent and c be the control/constant:
	
	\begin{center}
	\[y= f(x) + c\]
	\end{center}
	
	By keeping c constant and changing x we can derive $f(x)$ which is what many experiments try to do.
	Determining the relationship between the independent and the dependent variable.
	
	Lastly you should be able to evaluate the experimental method to determine whether or not it is appropriate to find the expected outcomes.
	This part requires critical thinking and an understanding of modules 3-4
	
	\paragraph{Implementing} is more about usage.
	I will not go into it fully here but here is an overview.
	
	You need to know your units, techniques and how to use the apparatus.
	Ask your teacher to show you the techniques and how to use the apparatus.
	The units will be visited as we explore the rest of this spec.
	
\section{The reviewing 1.1.3 \& 1.1.4}
	
	\paragraph{Reviewing} is necessary, it comprises two parts, that of analysis and of evaluation.
	
	\paragraph{Analysis} simply the processing and interpretation of qualitative\footnote{Observed data e.g. colour and effervescence} data combined with an understanding of the maths skills involved to explore quantitative\footnote{Data that is measured with a numeric output e.g. temperature} data.
	Basic statistics skills required are as follows:
	
	\begin{center}
		\begin{equation}
		\mean{x} = \frac{\sum_{i=1}^n x_i}{n}
		\end{equation}
		\textit{Technically not necessary but makes your PAGs feel more exact:}
		\begin{equation}
		y=a + bx \newline
		b=\frac{n\sum_{i=1}^n x_iy_i - \sum_{i=1}^n x_i \sum_{i=1}^n y_i}{\sum_{i=1}^n x^2_i -(\sum_{i=1}^n x_i)^2} \newline
		a = \mean{y} - b\mean{x}
		\end{equation}
	\end{center}
	
	Plotting graphs is important because this exam uses mostly graphical methods to find things such as gradients (which, at a point, it the gradient of the tangent to the curve often expressed as $\frac{dy}{dx}$).
	Just remember to use a good scale when drawing them.
	
	\paragraph{Evaluation} is the process in which we look back at the plan and the method and, well, evaluate it.
	You need to understand the limits of the procedure used, identifying anomalies and refine the experimental design by suggesting improvements to the procedures and apprentice.
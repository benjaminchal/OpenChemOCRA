\chapter{Organic chemistry and analysis}
`This module introduces several new functional groups
and emphasises the importance of organic synthesis.
This module also adds NMR spectroscopy to the instrumentation techniques used in organic and forensic analysis.'

\section{Aromatic Compounds 6.1.1}
	\paragraph{Benzene} has had two models, the evidence for each we will later examine, they are as follows:
	\begin{center}
	\chemfig{*6(-=-=-=)}

	\textit{Kekul\'{e} model of benzene}
	
	\vspace{7mm}	
	
	\chemfig{**6(------)}

	\textit{Benzene delocalised model}
	\end{center}
	
	\paragraph{The Kekul\'{e} model} of benzene hypotisises that there are 3 $\pi$ bonds and 6 $\sigma$ bonds.
	This was a conveniant way to describe the bonding in benzene but has, in recent years, been replaced by the more accurate delocalised model of benzene.
	
	\paragraph{The Delocalised model} of benzene is the one that is now generaly accepted. 
	It describes benzene's bonding in terms of a sideways overlap of the P orbitals (in the carbon atoms) above and bellow the plain, in a delocalised `system' of $\pi$ bonds.
	
	\paragraph{The evidence againced the Kekul\'{e} model,}
	\begin{enumerate}
		\item The low reactivity of benzene suggests that there are no double bonds present.
		The C=C bond present should decolourise bromine, however benzene dosn't decolourise bromine.
		
		\item X-ray defraction 
	\end{enumerate}